\chapter{One-dimensional Integrals in Several Variables} \label{path:chapter}

%%%%%%%%%%%%%%%%%%%%%%%%%%%%%%%%%%%%%%%%%%%%%%%%%%%%%%%%%%%%%%%%%%%%%%%%%%%%%%

%\sectionnewpage
\section{Differentiation under the integral}
\label{sec:diffunderint}

\sectionnotes{less than 1 lecture}

Let $f(x,y)$ be a function of two variables and define
\begin{equation*}
g(y) := \int_a^b f(x,y) ~dx .
\end{equation*}
Suppose $f$ is differentiable in $y$.  The question we ask is
when can we ``differentiate under the integral'', that is,
when is it true that $g$ is differentiable and its derivative
\begin{equation*}
g'(y) \overset{?}{=} \int_a^b \frac{\partial f}{\partial y}(x,y) ~dx .
\end{equation*}
Differentiation is a limit and therefore we are really asking when do the
two limiting operations of integration and differentiation commute.
As we have seen, this is not always possible, some sort of uniformity is
necessary.  In particular, the first
question we would face is the integrability of
$\frac{\partial f}{\partial y}$, but the formula can fail even if
$\frac{\partial f}{\partial y}$ is integrable for all $y$.

Let us prove a simple, but the most useful version of this theorem.

\begin{thm}[\myindex{Leibniz integral rule}]
Suppose $f \colon [a,b] \times [c,d] \to \R$ is a continuous function,
such that $\frac{\partial f}{\partial y}$ exists for all $(x,y) \in [a,b]
\times [c,d]$ and is continuous.  Define
\begin{equation*}
g(y) := \int_a^b f(x,y) ~dx .
\end{equation*}
Then $g \colon [c,d] \to \R$ is differentiable and
\begin{equation*}
g'(y) = \int_a^b \frac{\partial f}{\partial y}(x,y) ~dx .
\end{equation*}
\end{thm}

The continuity requirements for $f$ and
$\frac{\partial f}{\partial y}$ can be
weakened, but not dropped outright.  The main point is for
$\frac{\partial f}{\partial y}$ to exist and be continuous for a small
interval in the $y$ direction.  In applications, the $[c,d]$ can be a
small interval around the point where you need to differentiate.

\begin{proof}
Fix $y \in [c,d]$ and let $\epsilon > 0$ be given.
As $\frac{\partial f}{\partial y}$ is continuous on $[a,b] \times [c,d]$ it
is uniformly continuous.  In particular, there exists $\delta > 0$ such that
whenever $y_1 \in [c,d]$ with
$\abs{y_1-y} < \delta$ and all $x \in [a,b]$ we have
\begin{equation*}
\abs{\frac{\partial f}{\partial y}(x,y_1)-\frac{\partial f}{\partial y}(x,y)} < \epsilon .
\end{equation*}

Suppose $h$ is such that $y+h \in [c,d]$ and $\abs{h} < \delta$.
Fix $x$ for a moment
and apply mean value theorem to find a $y_1$ between $y$ and $y+h$ such that
\begin{equation*}
\frac{f(x,y+h)-f(x,y)}{h}
=
\frac{\partial f}{\partial y}(x,y_1) .
\end{equation*}
If $\abs{h} < \delta$, then
\begin{equation*}
\abs{
\frac{f(x,y+h)-f(x,y)}{h}
-
\frac{\partial f}{\partial y}(x,y) 
}
=
\abs{
\frac{\partial f}{\partial y}(x,y_1) 
-
\frac{\partial f}{\partial y}(x,y) 
}
< \epsilon .
\end{equation*}
This argument worked for every $x \in [a,b]$.  Therefore, as a function of
$x$
\begin{equation*}
x \mapsto \frac{f(x,y+h)-f(x,y)}{h}
\qquad
\text{converges uniformly to}
\qquad
x \mapsto \frac{\partial f}{\partial y}(x,y)
\qquad
\text{as $h \to 0$} .
\end{equation*}
We only defined uniform convergence for sequences although the idea is the
same.  If you wish you can replace $h$ with $\nicefrac{1}{n}$ above and let
$n \to \infty$.

Now consider the difference quotient
\begin{equation*}
\frac{g(y+h)-g(y)}{h}
=
\frac{\int_a^b f(x,y+h) ~dx -
\int_a^b f(x,y) ~dx }{h}
=
\int_a^b \frac{f(x,y+h)-f(x,y)}{h} ~dx .
\end{equation*}
Uniform convergence can be taken underneath the integral and therefore
\begin{equation*}
\lim_{h\to 0}
\frac{g(y+h)-g(y)}{h}
= 
\int_a^b 
\lim_{h\to 0}
\frac{f(x,y+h)-f(x,y)}{h} ~dx 
=
\int_a^b 
\frac{\partial f}{\partial y}(x,y) ~dx . \qedhere
\end{equation*}
\end{proof}

\begin{example}
Let
\begin{equation*}
f(y) = \int_0^1 \sin(x^2-y^2) ~dx .
\end{equation*}
Then
\begin{equation*}
f'(y) = \int_0^1 -2y\cos(x^2-y^2) ~dx .
\end{equation*}
\end{example}

\begin{example}
Suppose we start with
\begin{equation*}
\int_0^{1} \frac{x-1}{\ln(x)} ~dx .
\end{equation*}
The function under the integral 
extends to be continuous on $[0,1]$, and hence
the integral exists, see exercise below.  Trouble is finding it.  Introduce a parameter $y$
and define a function:
\begin{equation*}
g(y) := \int_0^{1} \frac{x^y-1}{\ln(x)} ~dx .
\end{equation*}
The function
$\frac{x^y-1}{\ln(x)}$
also extends to a continuous function of $x$ and $y$
for $(x,y) \in [0,1] \times [0,1]$.
Therefore
$g$ is a continuous function of on $[0,1]$.  In particular, $g(0) = 0$.
For any $\epsilon > 0$, the $y$ derivative of the integrand, $x^y$,
is continuous on $[0,1] \times [\epsilon,1]$.  Therefore,
for $y >0$ we may differentiate under the integral sign
\begin{equation*}
g'(y) =
\int_0^{1} \frac{\ln(x) x^y}{\ln(x)} ~dx 
=
\int_0^{1} x^y ~dx =
\frac{1}{y+1} .
\end{equation*}
We need to figure out $g(1)$, knowing $g'(y) = \frac{1}{y+1}$ and $g(0) =
0$.  By elementary calculus we find $g(1) = \int_0^1 g'(y)~dy = \ln(2)$.  Therefore
\begin{equation*}
\int_0^{1} \frac{x-1}{\ln(x)} ~dx  = \ln(2).
\end{equation*}
\end{example}

\begin{exercise}
Prove the two statements that were asserted in the example.\\
a) Prove $\frac{x-1}{\ln(x)}$ extends to a continuous function of
$[0,1]$.  That is, there exists a continuous function on $[0,1]$
that equals $\frac{x-1}{\ln(x)}$ on $(0,1)$.\\
b) Prove $\frac{x^y-1}{\ln(x)}$ extends to a continuous function
on $[0,1] \times [0,1]$.
\end{exercise}

\subsection{Exercises}

\begin{exercise}
Suppose $h \colon \R \to \R$ is a continuous function.  Suppose $g
\colon \R \to \R$ is which is continuously differentiable and compactly
supported.  That is there exists some $M > 0$ such that $g(x) = 0$ whenever
$\abs{x} \geq M$.  Define
\begin{equation*}
f(x) := \int_{-\infty}^\infty h(y)g(x-y)~dy  .
\end{equation*}
Show that $f$ is differentiable.
\end{exercise}

\begin{exercise}
Suppose $f \colon \R \to \R$ is an infinitely differentiable function (all derivatives exist)
such that $f(0) = 0$.  Then show that there exists another infinitely
differentiable function $g(x)$ such that $f(x) = xg(x)$.  Finally show that
if $f'(0) \not= 0$, then $g(0) \not= 0$.  Hint: first write
$f(x) = \int_0^x f'(s) ds$ and then rewrite the integral to go from $0$ to
1.
\end{exercise}


\begin{exercise}
Compute $\int_0^1 e^{tx} ~dx$.  Derive the formula for
$\int_0^1 x^n e^{x} ~dx$ not using integration by parts, but
by differentiation underneath the integral.
\end{exercise}

\begin{exercise}
Let $U \subset \R^n$ be an open set and suppose
$f(x,y_1,y_2,\ldots,y_n)$ is a continuous
function defined on $[0,1] \times U \subset \R^{n+1}$.
Suppose
$\frac{\partial f}{\partial y_1},
\frac{\partial f}{\partial y_2},\ldots,
\frac{\partial f}{\partial y_n}$
exist and are continuous on $[0,1] \times U$.
Then prove that $F \colon U \to \R$ defined by
\begin{equation*}
F(y_1,y_2,\ldots,y_n) :=
\int_0^1
f(x,y_1,y_2,\ldots,y_n)
\, dx
\end{equation*}
is continuously differentiable.
\end{exercise}

\begin{exercise}
Work out the following counterexample:  Let
\begin{equation*}
f(x,y) :=
\begin{cases}
\frac{xy^3}{{(x^2+y^2)}^2} & \text{if $x\not=0$ or $y\not= 0$,}\\
0 & \text{if $x=0$ and $y=0$.}
\end{cases}
\end{equation*}
a) Prove that for any fixed $y$ the function $x \mapsto f(x,y)$ is
Riemann integrable on $[0,1]$ and
\begin{equation*}
g(y) = \int_0^1 f(x,y) \, dx = \frac{y}{2y^2+2} .
\end{equation*}
Therefore $g'(y)$ exists and we get the continuous function
\begin{equation*}
g'(y) = \frac{1-y^2}{2{(y^2+1)}^2} .
\end{equation*}
b) Prove $\frac{\partial f}{\partial y}$ exists at all $x$ and $y$ and
compute it.\\
c) Show that for all $y$
\begin{equation*}
\int_0^1 \frac{\partial f}{\partial y} (x,y) \, dx
\end{equation*}
exists but
\begin{equation*}
g'(0) \not= \int_0^1 \frac{\partial f}{\partial y} (x,0) \, dx .
\end{equation*}
\end{exercise}

\begin{exercise}
Work out the following counterexample:  Let
\begin{equation*}
f(x,y) :=
\begin{cases}
xy^2 \sin\bigl(\frac{1}{x^3y}\bigr) & \text{if $x\not=0$ and $y\not= 0$,}\\
0 & \text{if $x=0$ or $y=0$.}
\end{cases}
\end{equation*}
a) Prove $f$ is continuous on $[0,1] \times [a,b]$ for any interval
$[a,b]$. Therefore the following function is well defined on $[a,b]$
\begin{equation*}
g(y) = \int_0^1 f(x,y) \, dx .
\end{equation*}
b) Prove $\frac{\partial f}{\partial y}$ exists for all $(x,y)$ in
$[0,1] \times [a,b]$, but is not continuous.
\\
c) Show that $\int_0^1 \frac{\partial f}{\partial y}(x,y) \, dx$ does not
exist if $y \not= 0$ even if we take improper integrals.
\end{exercise}

%\begin{exercise}
%Suppose $f \colon \R^2 \to \R$ is a continuously differentiable function.
%Define
%\begin{equation*}
%F(x,y) := \int_0^1 \bigl( x f(tx,ty) + y g(tx,ty) \bigr) \,dt .
%\end{equation*}
%
%F_x(x,y) := \int_0^1 f(tx,ty) +  tx f_x (tx,ty) + ty g_x(tx,ty) \,dt .
%
%F_y(x,y) := \int_0^1 t f_y (tx,ty)\,dt .
%
%
%
%\end{exercise}

%FIXME

%%%%%%%%%%%%%%%%%%%%%%%%%%%%%%%%%%%%%%%%%%%%%%%%%%%%%%%%%%%%%%%%%%%%%%%%%%%%%%

\sectionnewpage
\section{Path integrals}
\label{sec:pathintegral}

\sectionnotes{2--3 lectures}

\subsection{Piecewise smooth paths}

\begin{defn}
A continuously differentiable function $\gamma \colon [a,b] \to \R^n$ is
called a \emph{\myindex{smooth path}}
or a
\emph{\myindex{continuously differentiable path}}\footnote{The
word ``smooth'' is used
sometimes for continuously differentiable and sometimes for infinitely
differentiable functions in the literature.}
if
$\gamma$ is continuously differentiable and
$\gamma^{\:\prime}(t) \not= 0$ for all $t \in [a,b]$.

The function $\gamma$ is called a
\emph{\myindex{piecewise smooth path}} or a
\emph{\myindex{piecewise continuously differentiable path}}
if there exist finitely many points
$t_0 = a < t_1 < t_2 < \cdots < t_k = b$ such that
the restriction of the function $\gamma|_{[t_{j-1},t_j]}$ is smooth path.

We say $\gamma$ is a \emph{\myindex{simple path}} if $\gamma|_{(a,b)}$ is
a one-to-one function.  A $\gamma$ is 
a \emph{\myindex{closed path}} if $\gamma(a) = \gamma(b)$, that is
if the path starts and ends in the same point.
\end{defn}

Since $\gamma$ is a function of one variable,
we have seen before that treating $\gamma^{\:\prime}(t)$ as a matrix is equivalent
to treating it as a vector since it is an $n \times 1$ matrix, that is,
a column vector.
In fact, by an earlier exercise, even the operator norm of
$\gamma^{\:\prime}(t)$ is equal to
the euclidean norm.  Therefore, we will write $\gamma^{\:\prime}(t)$ as a vector
as is usual, and then $\gamma^{\:\prime}(t)$ is just the vector of the derivatives
of its components, so if $\gamma(t) =
\bigl( \gamma_1(t), \gamma_2(t), \ldots, \gamma_n(t) \bigr)$, then
$\gamma^{\:\prime}(t) =
\bigl( \gamma_1^{\:\prime}(t), \gamma_2^{\:\prime}(t), \ldots,
\gamma_n^{\:\prime}(t) \bigr)$.


One can often get by with only smooth paths, but for computations, the simplest
paths to write down are often piecewise smooth.
Note that a piecewise smooth function (or path) is automatically continuous (exercise).

Generally, it is the direct image $\gamma\bigl([a,b]\bigr)$ that is what we are
interested in, although how we parametrize it with $\gamma$ is also
important to some degree.  We informally talk about a curve, and often
we really mean the set $\gamma\bigl([a,b]\bigr)$, just as before depending
on context.
%defined the word curve to be any function $\gamma \colon [a,b] \to \R^n$
%without the derivative condition.

\begin{example} \label{mv:example:unitsquarepath}
Let $\gamma \colon [0,4] \to \R^2$ be defined by
\begin{equation*}
\gamma(t) :=
\begin{cases}
(t,0) & \text{if $t \in [0,1]$,}\\
(1,t-1) & \text{if $t \in (1,2]$,}\\
(3-t,1) & \text{if $t \in (2,3]$,}\\
(0,4-t) & \text{if $t \in (3,4]$.}
\end{cases}
\end{equation*}
Then the reader can check that the path is the unit square traversed
counterclockwise.  We can check that for example
$\gamma|_{[1,2]}(t) = (1,t-1)$ and therefore
$(\gamma|_{[1,2]})'(t) = (0,1) \not= 0$.  It is good to notice at this point
that
$(\gamma|_{[1,2]})'(1) = (0,1)$,
$(\gamma|_{[0,1]})'(1) = (1,0)$, and
$\gamma^{\:\prime}(1)$ does not exist.  That is, at the corners $\gamma$ is of course
not differentiable, even though the restrictions are differentiable and the
derivative depends on which restriction you take.
\end{example}

\begin{example}
The condition that $\gamma^{\:\prime}(t) \not= 0$ means that the image of $\gamma$
has no ``corners'' where $\gamma$ is continuously differentiable.  For example,
take the function
\begin{equation*}
\gamma(t) :=
\begin{cases}
(t^2,0) & \text{ if $t < 0$,}\\
(0,t^2) & \text{ if $t \geq 0$.}
\end{cases}
\end{equation*}
It is left for the reader to check that $\gamma$ is continuously
differentiable, yet the image $\gamma(\R) = \{ (x,y) \in \R^2 : (x,y) =
(s,0) \text{ or } (x,y) = (0,s) \text{ for some $s \geq 0$ } \}$ has a
``corner'' at the origin.  And that is because $\gamma^{\:\prime}(0) = (0,0)$.
More complicated examples with even infinitely many corners exist,
see the exercises.
\end{example}

The condition that $\gamma^{\:\prime}(t) \not= 0$ even at the endpoints guarantees
not only no corners, but also that the path ends nicely, that is, can
extend a little bit past the endpoints.  Again, see the exercises.

\begin{example}
A graph of a continuously differentiable function $f \colon [a,b] \to \R$ is a smooth path.
That is, define $\gamma \colon [a,b] \to \R^2$ by
\begin{equation*}
\gamma(t) := \bigl(t,f(t)\bigr) .
\end{equation*}
Then $\gamma^{\:\prime}(t) = \bigl( 1 , f'(t) \bigr)$, which is never zero.

There are other ways of parametrizing the path.  That is, having a
different path with the same image.  For example,
the function that takes $t$ to 
$(1-t)a+tb$, takes the interval $[0,1]$ to $[a,b]$.  So let
$\alpha \colon [0,1] \to \R^2$ be defined by
\begin{equation*}
\alpha(t) := \bigl((1-t)a+tb,f((1-t)a+tb)\bigr) .
\end{equation*}
Then
$\alpha'(t) = \bigl( b-a , (b-a)f'((1-t)a+tb) \bigr)$, which is never zero.
Furthermore as sets $\alpha\bigl([0,1]\bigr) = \gamma\bigl([a,b]\bigr)
= \{ (x,y) \in \R^2 : x \in [a,b] \text{ and } f(x) = y \}$,
which is just the graph of $f$.
\end{example}

The last example leads us to a definition.

\begin{defn}
Let $\gamma \colon [a,b] \to \R^n$ be a smooth path and
$h \colon [c,d] \to [a,b]$ a continuously differentiable bijective function
such that $h'(t) \not= 0$ for all $t \in [c,d]$.  Then
the composition
$\gamma \circ h$ is called a
\emph{\myindex{smooth reparametrization}}\index{reparametrization}
of $\gamma$.

Let $\gamma$ be a piecewise smooth path,
and $h$ be a piecewise smooth bijective function.  Then
the composition
$\gamma \circ h$ is called a
\emph{\myindex{piecewise smooth reparametrization}} of $\gamma$.

If $h$ is strictly increasing, then $h$ is 
said to \emph{\myindex{preserve orientation}}.  If $h$ does not preserve
orientation, then $h$ is said to \emph{\myindex{reverse orientation}}.
\end{defn}

A reparametrization is another path for the same set.  That is,
$(\gamma \circ h)\bigl([c,d]\bigr) =
\gamma \bigl([a,b]\bigr)$.

Let us remark that for $h$, piecewise smooth means that there is
some partition $t_0 = c < t_1 < t_2 < \cdots < t_k = d$,
such that $h|_{[t_{j-1},t_j]}$ is continuously differentiable
and $(h|_{[t_{j-1},t_j]})'(t) \not= 0$ for all $t \in [t_{j-1},t_j]$.
Since $h$ is bijective, it is either strictly increasing or
strictly decreasing.  Therefore either $(h|_{[t_{j-1},t_j]})'(t) > 0$
for all $t$ or $(h|_{[t_{j-1},t_j]})'(t) < 0$ for all $t$.

%Let us remark that if $h$ is smooth since the function $h'$ is continuous and $h'(t) \not= 0$ for all $t
%\in [c,d]$, then if $h'(t) < 0$ for one $t \in [c,d]$, then $h'(t) < 0$ for
%all $t \in [c,d]$ by the intermediate value theorem.  That is, $h'(t)$ has
%the same sign at every $t \in [c,d]$.

\begin{prop} \label{prop:reparamapiecewisesmooth}
If $\gamma \colon [a,b] \to \R^n$ is a piecewise smooth path,
and $\gamma \circ h \colon [c,d] \to \R^n$ is
a piecewise smooth reparametrization, then $\gamma \circ h$
is a piecewise smooth path.
\end{prop}

\begin{proof}
Let us assume that $h$ preserves orientation, that is, $h$ is strictly
increasing.
If $h \colon [c,d] \to [a,b]$ gives a piecewise smooth reparametrization,
then for some partition
$r_0 = c < r_1 < r_2 < \cdots < r_\ell = d$, we have
$h|_{[t_{j-1},t_j]}$ is continuously differentiable with positive
derivative.

%, then 
%$h'(t)$ has the same sign for all $t \in [c,d]$.  It is a bijective
%mapping with a continuously differentiable inverse, it is either strictly
%increasing or strictly decreasing.

Let $t_0 = a < t_1 < t_2 < \cdots < t_k = b$ be the partition from the
definition of piecewise smooth for $\gamma$ together with the 
points $\{ h(r_0), h(r_1), h(r_2), \ldots, h(r_\ell) \}$.
Let $s_j := h^{-1}(t_j)$.  Then
$s_0 = c < s_1 < s_2 < \cdots < s_k = d$.
For $t \in [s_{j-1},s_j]$ notice that $h(t) \in [t_{j-1},t_j]$,
$h|_{[s_{j-1},s_j]}$ is continuously differentiable, and
$\varphi|_{[t_{j-1},t_j]}$ is also continuously differentiable.
Then
\begin{equation*}
(\gamma \circ h)|_{[s_{j-1},s_{j}]} (t)
=
\gamma|_{[t_{j-1},t_{j}]} \bigl( h|_{[s_{j-1},s_j]}(t) \bigr) .
\end{equation*}
The function 
$(\gamma \circ h)|_{[s_{j-1},s_{j}]}$ is therefore continuously
differentiable and
by the chain rule
\begin{equation*}
\bigl( (\gamma \circ h)|_{[s_{j-1},s_{j}]} \bigr) ' (t)
=
\bigl( \gamma|_{[t_{j-1},t_{j}]} \bigr)' \bigl( h(t) \bigr)
(h|_{[s_{j-1},s_j]})'(t) \not= 0 .
\end{equation*}
Therefore $\gamma \circ h$ is a piecewise smooth path.  The case for an
orientation reversing $h$ is left as an exercise.
\end{proof}

%One need not have the reparametrization be smooth at all points, it really
%only needs to be again ``piecewise smooth,'' but the above definition will
%suffice for us and it keeps matters simpler, the generalization is left as
%an exercise.

%Furthermore,
If two paths are simple and their images are the same, it is
left as an exercise that there exists a reparametrization.

\subsection{Path integral of a one-form}

\begin{defn}
If $(x_1,x_2,\ldots,x_n) \in \R^n$ are our coordinates, and
given $n$ real-valued continuous functions $f_1,f_2,\ldots,f_n$ defined on some set $S \subset \R^n$ we
define a so-called \emph{\myindex{one-form}}:
\begin{equation*}
\omega = \omega_1 dx_1 + \omega_2 dx_2 + \cdots \omega_n dx_n .
\end{equation*}
We could represent $\omega$ as a continuous function from $S$ to $\R^n$, although it is
better to think of it as a different object.
\end{defn}

\begin{example}
For example,
\begin{equation*}
\omega(x,y) = \frac{-y}{x^2+y^2} dx + \frac{x}{x^2+y^2} dy
\end{equation*}
is a one-form defined on $\R^2 \setminus \{ (0,0) \}$.
\end{example}

\begin{defn}
Let $\gamma \colon [a,b] \to \R^n$ be a smooth path
and
\begin{equation*}
\omega = \omega_1 dx_1 + \omega_2 dx_2 + \cdots \omega_n dx_n ,
\end{equation*}
a one-form defined on the direct image $\gamma\bigl([a,b]\bigr)$.
Let $\gamma = (\gamma_1,\gamma_2,\ldots,\gamma_n)$ be the components of
$\gamma$.
Define:
\begin{equation*}
\begin{split}
\int_{\gamma} \omega
& :=
\int_a^b 
\Bigl(
\omega_1\bigl(\gamma(t)\bigr) \gamma_1^{\:\prime}(t) +
\omega_2\bigl(\gamma(t)\bigr) \gamma_2^{\:\prime}(t) + \cdots +
\omega_n\bigl(\gamma(t)\bigr) \gamma_n^{\:\prime}(t) \Bigr) \, dt
\\
&\phantom{:}=
\int_a^b 
\left(
\sum_{j=1}^n
\omega_j\bigl(\gamma(t)\bigr) \gamma_j^{\:\prime}(t) \right) \, dt .
\end{split}
\end{equation*}
If $\gamma$ is piecewise smooth, take the corresponding partition
$t_0 = a < t_1 < t_2 < \ldots < t_k = b$, where we assume the partition is
the minimal one, that is $\gamma$ is not differentiable
at $t_2,t_3,\ldots,t_{k-1}$.  Each $\gamma|_{[t_{j-1},t_j]}$ is
a smooth path and we define
\begin{equation*}
\int_{\gamma} \omega
:=
\int_{\gamma|_{[t_0,t_1]}} \omega
\,
+
\,
\int_{\gamma|_{[t_1,t_2]}} \omega
\,
+ \, \cdots \, + \,
\int_{\gamma|_{[t_{n-1},t_n]}} \omega .
\end{equation*}
\end{defn}

The notation makes sense from the formula you remember from calculus,
let us state it somewhat informally:
if $x_j(t) = \gamma_j(t)$, then $dx_j = \gamma_j^{\:\prime}(t) dt$.

Paths can be cut up or concatenated as follows.  The proof is a direct application
of the additivity of the Riemann integral, and is left as an exercise.
The proposition also justifies why we defined the integral over a piecewise
smooth path in the way we did, and it further justifies that we may as well
have taken any partition not just the minimal one in the definition.

\begin{prop} \label{mv:prop:pathconcat}
Let $\gamma \colon [a,c] \to \R^n$ be a piecewise smooth path.
For some $b \in (a,c)$,
define the piecewise smooth paths
$\alpha = \gamma|_{[a,b]}$ and
$\beta = \gamma|_{[b,c]}$.
For a one-form $\omega$ defined on the image
of $\gamma$ we have
\begin{equation*}
\int_{\gamma} \omega =
\int_{\alpha} \omega +
\int_{\beta} \omega .
\end{equation*}
\end{prop}


\begin{example} \label{example:mv:irrotoneformint}
Let the one-form $\omega$ and the path $\gamma \colon [0,2\pi] \to \R^2$ be defined by
\begin{equation*}
\omega(x,y) := \frac{-y}{x^2+y^2} dx + \frac{x}{x^2+y^2} dy,
\qquad
\gamma(t) := \bigl(\cos(t),\sin(t)\bigr) .
\end{equation*}
Then
\begin{equation*}
\begin{split}
\int_{\gamma} \omega
& =
\int_0^{2\pi}
\Biggl(
\frac{-\sin(t)}{{\bigl(\cos(t)\bigr)}^2+{\bigl(\sin(t)\bigr)}^2}
\bigl(-\sin(t)\bigr)
+
\frac{\cos(t)}{{\bigl(\cos(t)\bigr)}^2+{\bigl(\sin(t)\bigr)}^2}
\bigl(\cos(t)\bigr)
\Biggr) \, dt
\\
& =
\int_0^{2\pi}
1 \, dt
= 2\pi .
\end{split}
\end{equation*}
Next, let us parametrize the same curve as
$\alpha \colon [0,1] \to \R^2$ defined by $\alpha(t) := \bigl(\cos(2\pi
t),\sin(2 \pi t)\bigr)$, that is $\alpha$ is a smooth reparametrization of
$\gamma$.  Then
\begin{equation*}
\begin{split}
\int_{\alpha} \omega
& =
\int_0^{1}
\Biggl(
\frac{-\sin(2\pi t)}{{\bigl(\cos(2\pi t)\bigr)}^2+{\bigl(\sin(2\pi t)\bigr)}^2}
\bigl(-2\pi \sin(2\pi t)\bigr)
\\
& \phantom{=\int_0^1\Biggl(~}
+
\frac{\cos(2 \pi t)}{{\bigl(\cos(2 \pi t)\bigr)}^2+{\bigl(\sin(2 \pi t)\bigr)}^2}
\bigl(2 \pi \cos(2 \pi t)\bigr)
\Biggr) \, dt
\\
& =
\int_0^{1}
2\pi \, dt
= 2\pi .
\end{split}
\end{equation*}
Now let us reparametrize with $\beta \colon [0,2\pi] \to \R^2$
as $\beta(t) := \bigl(\cos(-t),\sin(-t)\bigr)$.  Then
\begin{equation*}
\begin{split}
\int_{\beta} \omega
& =
\int_0^{2\pi}
\Biggl(
\frac{-\sin(-t)}{{\bigl(\cos(-t)\bigr)}^2+{\bigl(\sin(-t)\bigr)}^2}
\bigl(\sin(-t)\bigr)
+
\frac{\cos(-t)}{{\bigl(\cos(-t)\bigr)}^2+{\bigl(\sin(-t)\bigr)}^2}
\bigl(-\cos(-t)\bigr)
\Biggr) \, dt
\\
& =
\int_0^{2\pi}
(-1) \, dt
= -2\pi .
\end{split}
\end{equation*}
Now, $\alpha$ was an orientation preserving reparametrization of
$\gamma$, and the integral was the same.  On the other hand $\beta$
is an orientation reversing reparametrization and the integral was
minus the original.
\end{example}

The previous example is not a fluke.
The path integral does not depend on the parametrization of
the curve, the only thing that matters is the direction in which the curve
is traversed.

\begin{prop} \label{mv:prop:pathintrepararam}
Let $\gamma \colon [a,b] \to \R^n$ be a piecewise smooth path and
$\gamma \circ h \colon [c,d] \to \R^n$ a piecewise smooth reparametrization.
Suppose $\omega$ is a one-form defined on the set $\gamma\bigl([a,b]\bigr)$.  Then
\begin{equation*}
\int_{\gamma \circ h} \omega =
\begin{cases}
\int_{\gamma} \omega & \text{ if $h$ preserves orientation,}\\
-\int_{\gamma} \omega & \text{ if $h$ reverses orientation.}
\end{cases}
\end{equation*}
\end{prop}

\begin{proof}
Assume first that $\gamma$ and $h$ are both smooth.
Write the one-form  as $\omega = \omega_1 dx_1 + \omega_2 dx_2 + \cdots +
\omega_n dx_n$.
Suppose first that $h$ is orientation preserving.  Using
the definition of the path integral and the change of variables
formula for the Riemann integral,
\begin{equation*}
\begin{split}
\int_{\gamma} \omega
& =
\int_a^b 
\left(
\sum_{j=1}^n
\omega_j\bigl(\gamma(t)\bigr) \gamma_j^{\:\prime}(t)
\right) \, dt
%\left(
%\omega_1\bigl(\gamma(t)\bigr) \gamma_1^{\:\prime}(t) +
%\omega_2\bigl(\gamma(t)\bigr) \gamma_2^{\:\prime}(t) + \cdots +
%\omega_n\bigl(\gamma(t)\bigr) \gamma_n^{\:\prime}(t) \right) \, dt 
\\
& =
\int_c^d 
\left(
\sum_{j=1}^n
\omega_j\Bigl(\gamma\bigl(h(\tau)\bigr)\Bigr) \gamma_j^{\:\prime}\bigl(h(\tau)\bigr)
\right) h'(\tau) \, d\tau
%\left(
%\omega_1\bigl(\gamma(h(\tau))\bigr) \gamma_1^{\:\prime}(h(\tau)) +
%\omega_2\bigl(\gamma(h(\tau))\bigr) \gamma_2^{\:\prime}(h(\tau)) + \cdots +
%\omega_n\bigl(\gamma(h(\tau))\bigr) \gamma_n^{\:\prime}(h(\tau)) \right) h'(\tau) \, d\tau 
\\
& =
\int_c^d 
\left(
\sum_{j=1}^n
\omega_j\Bigl(\gamma\bigl(h(\tau)\bigr)\Bigr) (\gamma_j \circ h)'(\tau)
\right) \, d\tau
%\left(
%\omega_1\bigl(\gamma(h(\tau))\bigr) (\gamma_1 \circ h)'(\tau) +
%\omega_2\bigl(\gamma(h(\tau))\bigr) (\gamma_2 \circ h)'(\tau) + \cdots +
%\omega_n\bigl(\gamma(h(\tau))\bigr) (\gamma_n \circ h)'(\tau) \right) \, d\tau 
%\\
%& = 
=
\int_{\gamma \circ h} \omega .
\end{split}
\end{equation*}
If $h$ is orientation reversing it will swap the order of the limits on the
integral introducing a minus sign.  The details, along with finishing the proof for piecewise smooth
paths is left to the reader as \exerciseref{mv:exercise:pathpiece}.
\end{proof}

Due to this proposition (and the exercises), if we have a set $\Gamma
\subset \R^n$ that is the image of a simple piecewise smooth path
$\gamma\bigl([a,b]\bigr)$, then if we somehow indicate the orientation, that
is, which direction we traverse the curve, in other words where we start and
where we finish. Then we just write
\begin{equation*}
\int_{\Gamma} \omega ,
\end{equation*}
without mentioning the specific $\gamma$.
Furthermore, for a simple closed path, it does not even matter where we
start the parametrization.  See the exercises.

Recall that \emph{simple} means that $\gamma$ restricted to $(a,b)$ is
one-to-one, that is, it is one-to-one except perhaps at the endpoints.
We also often relax the simple path condition a little bit.
For example, as long as
$\gamma \colon [a,b] \to \R^n$ is one-to-one except at finitely many points.  That
is, there are only finitely many points $p \in \R^n$ such that
$\gamma^{-1}(p)$ is more than one point.  See the exercises.  The issue about the
injectivity
problem is illustrated by the following example.

\begin{example}
Suppose $\gamma \colon [0,2\pi] \to \R^2$ is given by $\gamma(t) :=
\bigl(\cos(t),\sin(t)\bigr)$ and
$\beta \colon [0,2\pi] \to \R^2$ is given by $\beta(t) :=
\bigl(\cos(2t),\sin(2t)\bigr)$.  Notice that
$\gamma\bigl([0,2\pi]\bigr) = \beta\bigl([0,2\pi]\bigr)$, and we travel
around the same curve, the unit circle.  But $\gamma$ goes around the unit
circle once in the counter clockwise direction, and $\beta$ goes around the
unit circle twice (in the same direction).  Then
\begin{align*}
& \int_{\gamma} -y\, dx + x\,dy
=
\int_0^{2\pi}
\Bigl( \bigl(-\sin(t) \bigr) \bigl(-\sin(t) \bigr) + \cos(t) \cos(t) \Bigr) dt
=
2 \pi,\\
& \int_{\beta} -y\, dx + x\,dy
=
\int_0^{2\pi}
\Bigl( \bigl(-\sin(2t) \bigr) \bigl(-2\sin(2t) \bigr) + \cos(t)
\bigl(2\cos(t)\bigr) \Bigr) dt
=
4 \pi.
\end{align*}
\end{example}

It is sometimes convenient to define a path integral over $\gamma \colon
[a,b] \to \R^n$ that is not a path.
We define
\begin{equation*}
\int_{\gamma} \omega := \int_a^b
\left(
\sum_{j=1}^n
\omega_j\bigl(\gamma(t)\bigr) \gamma_j^{\:\prime}(t)
\right) \, dt 
\end{equation*}
for any $\gamma$ which is continuously differentiable.  A 
case which comes up naturally is when $\gamma$ is constant.  In this case
$\gamma^{\:\prime}(t) = 0$ for all $t$ and $\gamma\bigl([a,b]\bigr)$ is a single
point, which we regard as a ``curve'' of length zero.  Then,
$\int_{\gamma} \omega = 0$.


%We end with a definition.  A path $\gamma \colon [a,b] \to \R^n$ is
%\emph{closed}\index{closed path}, if $\gamma(a) = \gamma(b)$.  It turns out
%that for a closed path, it does not matter where the point is that we start
%and end with is.  It only matters on the direction in which we traverse the
%path.  See the exercises.

%We end with a definition.  A path $\gamma \colon [a,b] \to \R^n$ is
%\emph{closed}\index{closed path}, if $\gamma(a) = \gamma(b)$.  It turns out
%that for a closed path, it does not matter where the point is that we start
%and end with is.  It only matters on the direction in which we traverse the
%path.  See the exercises.

\subsection{Line integral of a function}

Sometimes we wish to simply integrate a function against the so-called
arc-length measure.

\begin{defn}
Suppose $\gamma \colon [a,b] \to \R^n$ is a smooth path, and $f$ is a
continuous function defined on the image $\gamma\bigl([a,b]\bigr)$.  Then
define
\begin{equation*}
\int_{\gamma} f \,ds :=
\int_a^b f\bigl( \gamma(t) \bigr) \snorm{\gamma^{\:\prime}(t)} \, dt .
\end{equation*}

The definition for a piecewise smooth path is similar as before and is left
to the reader.
\end{defn}

The geometric idea of this integral is to find the ``area under the
graph of a function'' as we move around the path $\gamma$.
The line integral of a function is also independent of the parametrization,
and in this case, the orientation does not matter.

\begin{prop} \label{mv:prop:lineintrepararam}
Let $\gamma \colon [a,b] \to \R^n$ be a piecewise smooth path and
$\gamma \circ h \colon [c,d] \to \R^n$ a piecewise smooth reparametrization.
Suppose $f$ is a continuous function defined on the set
$\gamma\bigl([a,b]\bigr)$.  Then
\begin{equation*}
\int_{\gamma \circ h} f\, ds = \int_{\gamma} f\, ds .
\end{equation*}
\end{prop}

\begin{proof}
Suppose first that $h$ is orientation preserving and $\gamma$ and $h$
are both smooth.  Then as before
\begin{equation*}
\begin{split}
\int_{\gamma} f \, ds
& =
\int_a^b 
f\bigl(\gamma(t)\bigr) \snorm{\gamma^{\:\prime}(t)} \, dt
\\
& =
\int_c^d 
f\Bigl(\gamma\bigl(h(\tau)\bigr)\Bigr) \snorm{\gamma^{\:\prime}\bigl(h(\tau)\bigr)} h'(\tau) \, d\tau
\\
& =
\int_c^d 
f\Bigl(\gamma\bigl(h(\tau)\bigr)\Bigr) \snorm{\gamma^{\:\prime}\bigl(h(\tau)\bigr) h'(\tau)} \, d\tau
\\
& =
\int_c^d 
f\bigl((\gamma \circ h)(\tau)\bigr) \snorm{(\gamma \circ h)'(\tau)} \, d\tau
\\
& = 
\int_{\gamma \circ h} f \, ds .
\end{split}
\end{equation*}
If $h$ is orientation reversing it will swap the order of the limits on the
integral but you also have to introduce a minus sign in order
to take $h'$ inside the norm.
The details, along with finishing the proof for piecewise smooth
paths is left to the reader as \exerciseref{mv:exercise:linepiece}.
\end{proof}

Similarly as before, because of this proposition (and the exercises),
if $\gamma$ is simple, it does not matter which
parametrization we use.  Therefore, if $\Gamma = \gamma\bigl( [a,b] \bigr)$ we can
simply write
\begin{equation*}
\int_\Gamma f\, ds .
\end{equation*}
In this case we also do not need to worry about orientation, either way we
get the same thing.

\begin{example}
Let $f(x,y) = x$.  Let $C \subset \R^2$ be half of the unit circle for $x
\geq 0$.  We wish to compute
\begin{equation*}
\int_C f \, ds .
\end{equation*}
Parametrize the curve $C$ via $\gamma \colon
[\nicefrac{-\pi}{2},\nicefrac{\pi}{2}] \to \R^2$ defined as
$\gamma(t) := \bigl(\cos(t),\sin(t)\bigr)$.
Then $\gamma^{\:\prime}(t) = \bigl(-\sin(t),\cos(t)\bigr)$, and
\begin{equation*}
\int_C f \, ds =
\int_\gamma f \, ds
=
\int_{-\pi/2}^{\pi/2} \cos(t) \sqrt{ {\bigl(-\sin(t)\bigr)}^2 +  
{\bigl(\cos(t)\bigr)}^2 } \, dt
=
\int_{-\pi/2}^{\pi/2} \cos(t) \, dt = 2.
\end{equation*}
\end{example}

\begin{defn}
Suppose $\Gamma \subset \R^n$ is parametrized by a simple
piecewise smooth path $\gamma \colon [a,b] \to \R^n$, that is
$\gamma\bigl( [a,b] \bigr) = \Gamma$.  The we define the
\emph{\myindex{length}}\index{length of a curve} by
\begin{equation*}
\ell(\Gamma) := \int_{\Gamma} ds = \int_{\gamma} ds = \int_a^b
\snorm{\gamma^{\:\prime}(t)}\, dt .
\end{equation*}
\end{defn}

\begin{example}
Let $x,y \in \R^n$ be two points and write $[x,y]$ as the straight line
segment between the two points $x$ and $y$.  We parametrize
$[x,y]$ by $\gamma(t) := (1-t)x + ty$ for $t$ running between $0$ and $1$.
We find $\gamma^{\:\prime}(t) = y-x$ and therefore
\begin{equation*}
\ell\bigl([x,y]\bigr)
=
\int_{[x,y]} ds
=
\int_0^1 \snorm{y-x} \, dt
=
\snorm{y-x} .
\end{equation*}
So the length of $[x,y]$ is the distance between $x$ and $y$ in the
euclidean metric.
\end{example}

A simple piecewise smooth path $\gamma \colon [0,r] \to \R^n$ is
said to be an \emph{\myindex{arc-length parametrization}} if
\begin{equation*}
\ell\bigl( \gamma\bigl([0,t]\bigr) \bigr) = \int_0^t
\snorm{\gamma^{\:\prime}(\tau)}
\, d\tau  = t .
\end{equation*}
You can think of such a parametrization as moving around your curve at speed
1.

%FIXME:

\subsection{Exercises}

\begin{exercise}
Show that if $\varphi \colon [a,b] \to \R^n$ is piecewise smooth as we
defined it, then $\varphi$ is a continuous function.
\end{exercise}

\begin{exercise}
Finish the proof of \propref{prop:reparamapiecewisesmooth} for orientation
reversing reparametrizations.
\end{exercise}

\begin{exercise}
Prove \propref{mv:prop:pathconcat}.
\end{exercise}


%\begin{exercise}
%Show that if $h \colon [c,d] \to [a,b]$ is piecewise smooth bijective
%function (same definition as for paths, in fact you could think of it as a
%path into $\R$) and $\varphi \colon [a,b] \to \R^n$ is a piecewise smooth
%path, then $\varphi \circ h$ is also a piecewise smooth path.
%\end{exercise}

\begin{exercise} \label{mv:exercise:pathpiece}
Finish the proof of \propref{mv:prop:pathintrepararam}
for a)~orientation reversing reparametrizations, and b)~piecewise smooth paths
and reparametrizations.
\end{exercise}

\begin{exercise} \label{mv:exercise:linepiece}
Finish the proof of \propref{mv:prop:lineintrepararam}
for a)~orientation reversing reparametrizations, and b)~piecewise smooth paths
and reparametrizations.
\end{exercise}

\begin{exercise}
Suppose $\gamma \colon [a,b] \to \R^n$ is a piecewise smooth path, and $f$ is a
continuous function defined on the image $\gamma\bigl([a,b]\bigr)$.
Provide a definition of $\int_{\gamma} f \,ds$.
\end{exercise}

\begin{exercise}
Directly using the definitions compute:\\
a)~the arc-length of the unit square from
\exampleref{mv:example:unitsquarepath} using the given parametrization.
\\
b)~the arc-length of the unit circle using the parametrization
$\gamma \colon [0,1] \to \R^2$, $\gamma(t) := \bigl(\cos(2\pi t),\sin(2\pi t)\bigr)$.
\\
c)~the arc-length of the unit circle using the parametrization
$\beta \colon [0,2\pi] \to \R^2$, $\beta(t) := \bigl(\cos(t),\sin(t)\bigr)$.
\end{exercise}

\begin{exercise}
Suppose $\gamma \colon [0,1] \to \R^n$ is a smooth path, and
$\omega$ is a one-form defined on the image $\gamma\bigl([a,b]\bigr)$.
For $r \in [0,1]$, let $\gamma_r \colon [0,r] \to \R^n$ be defined
as simply the restriction of $\gamma$ to $[0,r]$.  Show that the
function $h(r) := \int_{\gamma_r} \omega$ is a continuously
differentiable function on $[0,1]$.
\end{exercise}

\begin{exercise}
Suppose $\gamma \colon [a,b] \to \R^n$ is a smooth path.
Show that there exists an $\epsilon > 0$ and a smooth function
$\tilde{\gamma} \colon (a-\epsilon,b+\epsilon) \to \R^n$
with $\tilde{\gamma}(t) = \gamma(t)$ for all $t \in [a,b]$
and $\tilde{\gamma}'(t) \not= 0$ for all $t \in 
(a-\epsilon,b+\epsilon)$.  That is, prove that a smooth path extends
some small distance past the end points.
\end{exercise}

\begin{exercise} 
Suppose $\alpha \colon [a,b] \to \R^n$ and
$\beta \colon [c,d] \to \R^n$ are piecewise smooth paths such that
$\Gamma := \alpha\bigl([a,b]\bigr) = \beta\bigl([c,d]\bigr)$.
Show that there exist finitely many points
$\{ p_1,p_2,\ldots,p_k\} \in \Gamma$, such that
the sets
$\alpha^{-1}\bigl( \{ p_1,p_2,\ldots,p_k\} \bigr)$
and
$\beta^{-1}\bigl( \{ p_1,p_2,\ldots,p_k\} \bigr)$
are partitions of $[a,b]$ and $[c,d]$, such that on any subinterval
the paths are smooth (that is, they are partitions as in the definition
of piecewise smooth path).
\end{exercise}

\begin{exercise}
a)~Suppose $\gamma \colon [a,b] \to \R^n$ and $\alpha \colon [c,d] \to \R^n$
are two smooth paths which are one-to-one and
$\gamma\bigl([a,b]\bigr) = \alpha\bigl([c,d]\bigr)$.  Then
there exists a smooth reparametrization $h \colon [a,b] \to [c,d]$
such that $\gamma = \alpha \circ h$.  Hint: It should be not hard to find
some $h$.  The trick is to show it is continuously differentiable
with a nonvanishing derivative.  You will want to apply the implicit function
theorem and it may at first seem the dimensions don't seem to work out.
\\
b) Prove the same thing as part a, but now for simple closed paths with the
further assumption that $\gamma(a) = \gamma(b) = \alpha(c) = \alpha(d)$.
\\
c) Prove parts a) and b) but for piecewise smooth paths, obtaining
piecewise smooth reparametrizations.  Hint: The trick is to find two
partitions such that when restricted to a subinterval of the partition
both paths have the same image and are smooth, see the above exercise.
\end{exercise}

\begin{exercise} 
Suppose $\alpha \colon [a,b] \to \R^n$ and
$\beta \colon [b,c] \to \R^n$ are piecewise smooth paths with
$\alpha(b)=\beta(b)$.  Let $\gamma \colon [a,c] \to \R^n$ be defined by
\begin{equation*}
\gamma(t) :=
\begin{cases}
\alpha(t) & \text{ if $t \in [a,b]$,} \\
\beta(t) & \text{ if $t \in (b,c]$.}
\end{cases}
\end{equation*}
Show that $\gamma$ is a piecewise smooth path, and that if $\omega$ is a
one-form defined on the curve given by $\gamma$, then
\begin{equation*}
\int_{\gamma} \omega =
\int_{\alpha} \omega +
\int_{\beta} \omega .
\end{equation*}
\end{exercise}

\begin{exercise} \label{mv:exercise:closedcurveintegral}
Suppose $\gamma \colon [a,b] \to \R^n$ and
$\beta \colon [c,d] \to \R^n$ are two simple piecewise smooth closed paths.
That is $\gamma(a)=\gamma(b)$ and $\beta(c) = \beta(d)$ and
the restrictions $\gamma|_{(a,b)}$ and $\beta|_{(c,d)}$ are one-to-one.
Suppose $\Gamma = \gamma\bigl([a,b]\bigr) = \beta\bigl([c,d]\bigr)$ and
$\omega$ is a one-form defined on $\Gamma \subset \R^n$.  Show that either
\begin{equation*}
\int_\gamma \omega = 
\int_\beta \omega,
\qquad \text{or} \qquad 
\int_\gamma \omega = 
- \int_\beta \omega.
\end{equation*}
In particular, the notation $\int_{\Gamma} \omega$ makes sense if we indicate
the direction in which the integral is evaluated.
Hint: see previous three exercises.
\end{exercise}

\begin{exercise} \label{mv:exercise:curveintegral}
Suppose $\gamma \colon [a,b] \to \R^n$ and
$\beta \colon [c,d] \to \R^n$ are two piecewise smooth paths
which are one-to-one except at finitely many points.  That is, there is at
most finitely many points $p \in \R^n$ such that
$\gamma^{-1}(p)$ or $\beta^{-1}(p)$ contains more than one point.
Suppose $\Gamma = \gamma\bigl([a,b]\bigr) = \beta\bigl([c,d]\bigr)$ and $\omega$ is a
one-form defined on $\Gamma \subset \R^n$.  Show that either
\begin{equation*}
\int_\gamma \omega = 
\int_\beta \omega,
\qquad \text{or} \qquad 
\int_\gamma \omega = 
- \int_\beta \omega.
\end{equation*}
In particular, the notation $\int_{\Gamma} \omega$ makes sense if we indicate
the direction in which the integral is evaluated.
\\
Hint: same hint as the last exercise.
\end{exercise}

\begin{samepage}
\begin{exercise}
Define $\gamma \colon [0,1] \to \R^2$ by
$\gamma(t) := \Bigl( t^3 \sin(\nicefrac{1}{t}),
t{\bigl(3t^2\sin(\nicefrac{1}{t})-t\cos(\nicefrac{1}{t})\bigr)}^2 \Bigr)$
for
$t \not= 0$ and $\gamma(0) = (0,0)$.  Show that:\\
a) $\gamma$ is continuously differentiable on $[0,1]$.\\
b) Show that there exists an infinite sequence $\{ t_n \}$ in $[0,1]$
converging to 0, such that
$\gamma^{\:\prime}(t_n) = (0,0)$. \\
c) Show that the points $\gamma(t_n)$ lie on the line $y=0$ and such
that the $x$-coordinate of $\gamma(t_n)$ alternates between positive and
negative (if they do not alternate you only found a subsequence
and you need to find them all).\\
d) Show that there is no piecewise smooth $\alpha$ whose image equals
$\gamma\bigl([0,1]\bigr)$.  Hint: look at part c) and show that $\alpha'$
must be zero where it reaches the origin.
\\
e) (Computer) if you know a plotting software that allows you to plot
parametric curves, make a plot of the curve, but only for $t$ in the
range $[0,0.1]$ otherwise you will not see the behavior.  In particular, you
should notice that $\gamma\bigl([0,1]\bigr)$ has infinitely many ``corners''
near the origin.
\end{exercise}
\end{samepage}

%%%%%%%%%%%%%%%%%%%%%%%%%%%%%%%%%%%%%%%%%%%%%%%%%%%%%%%%%%%%%%%%%%%%%%%%%%%%%%

\sectionnewpage
\section{Path independence}
\label{sec:pathind}

\sectionnotes{2 lectures}

\subsection{Path independent integrals}

Let $U \subset \R^n$ be a set and $\omega$ a one-form defined on $U$,
The integral of $\omega$
is said to be \emph{\myindex{path independent}}
if for any two points $x,y \in U$ and
any two piecewise smooth paths
$\gamma \colon [a,b] \to U$ and
$\beta \colon [c,d] \to U$ such that $\gamma(a) = \beta(c) = x$
and $\gamma(b) = \beta(d) = y$ we have
\begin{equation*}
\int_\gamma \omega = \int_\beta \omega .
\end{equation*}
In this case we simply write
\begin{equation*}
\int_x^y \omega := \int_\gamma \omega = \int_\beta \omega .
\end{equation*}
Not every one-form gives a path independent integral.  In fact, most do not.

\begin{example}
Let $\gamma \colon [0,1] \to \R^2$ be the path $\gamma(t) = (t,0)$
going from $(0,0)$ to $(1,0)$.  Let $\beta \colon [0,1] \to \R^2$ be the path
$\beta(t) = \bigl(t,(1-t)t\bigr)$ also going between the same points.  Then
\begin{align*}
& \int_\gamma y \, dx = 
\int_0^1 \gamma_2(t) \gamma_1^{\:\prime}(t) \, dt
=
\int_0^1 0 (1) \, dt = 0 ,\\
& \int_\beta y \, dx = 
\int_0^1 \beta_2(t) \beta_1'(t) \, dt
=
\int_0^1 (1-t)t(1) \, dt = \frac{1}{6} .
\end{align*}
So the integral of $y\,dx$ is not path independent.
In particular,
$\int_{(0,0)}^{(1,0)} y\,dx$ does not make sense.
\end{example}

\begin{defn}
Let $U \subset \R^n$ be an open set and $f \colon U \to \R$ a 
continuously differentiable function.  Then the one-form
\begin{equation*}
df :=
\frac{\partial f}{\partial x_1} \, dx_1 + 
\frac{\partial f}{\partial x_2} \, dx_2 + \cdots +
\frac{\partial f}{\partial x_n} \, dx_n 
\end{equation*}
is called the \emph{\myindex{total derivative}} of $f$.

An open set $U \subset \R^n$ is said to be \emph{\myindex{path connected}}%
\footnote{Normally only a continuous path is used in this definition, but
for open sets the two definitions are equivalent.  See the exercises.}
if for every two points $x$ and $y$ in $U$, there exists a piecewise smooth
path starting at $x$ and ending at $y$.
\end{defn}

We will leave as an exercise that every connected open set is path
connected.

\begin{prop} \label{mv:prop:pathinddf}
Let $U \subset \R^n$ be a path connected open set and $\omega$ a one-form
defined on $U$.  Then
\begin{equation*}
\int_x^y \omega
\end{equation*}
is path independent (for all $x,y \in U$) if and only if there exists
a continuously differentiable $f \colon U \to \R$ such that $\omega = df$.

In fact, if such an $f$ exists, then for any two points $x,y \in U$
\begin{equation*}
\int_{x}^y \omega = f(y)-f(x) .
\end{equation*}
\end{prop}

In other words if we fix $p \in U$, then $f(x) = C + \int_{p}^x \omega$.

\begin{proof}
First suppose that the integral is path independent.  Pick $p \in U$
and define
\begin{equation*}
f(x) := \int_{p}^x \omega .
\end{equation*}
Write $\omega = \omega_1 dx_1 + \omega_2 dx_2 + \cdots + \omega_n dx_n$.
We wish to show that for every $j = 1,2,\ldots,n$, the
partial derivative $\frac{\partial f}{\partial x_j}$ exists
and is equal to $\omega_j$.

Let $e_j$ be an arbitrary standard basis vector.  Compute
\begin{equation*}
\frac{f(x+h e_j) - f(x)}{h} =
\frac{1}{h} \left( \int_{p}^{x+he_j} \omega - \int_{p}^x \omega \right)
=
\frac{1}{h} \int_{x}^{x+he_j} \omega ,
\end{equation*}
which follows by \propref{mv:prop:pathconcat} and path indepdendence as 
$\int_{p}^{x+he_j} \omega =
\int_{p}^{x} \omega +
\int_{x}^{x+he_j} \omega$, because we could have picked a path from $p$ to
$x+he_j$ that also happens to pass through $x$, and then cut this path in
two.


Since $U$ is open, suppose $h$ is so small so that all points of distance
$\abs{h}$ or
less from $x$ are in $U$.
As the integral is path independent,
pick the simplest path possible from $x$ to $x+he_j$, that is
$\gamma(t) = x+t he_j$ for $t \in [0,1]$.  The path is in $U$.
Notice $\gamma^{\:\prime}(t) = h e_j$
has only one nonzero component and that is the $j$th component, which is
$h$.  Therefore
\begin{equation*}
\frac{1}{h} \int_{x}^{x+he_j} \omega 
=
\frac{1}{h} \int_{\gamma} \omega 
=
\frac{1}{h} \int_0^1 \omega_j(x+the_j) h \, dt 
=
\int_0^1 \omega_j(x+the_j) \, dt  .
\end{equation*}
We wish to take the limit as $h \to 0$.  The function $\omega_j$ is
continuous.  So given $\epsilon > 0$, $h$ can be small enough so that
$\abs{\omega(x)-\omega(y)} < \epsilon$, whenever $\snorm{x-y} \leq \abs{h}$.
Therefore,
$\abs{\omega_j(x+the_j)-\omega_j(x)} < \epsilon$ for all $t \in [0,1]$,
and we estimate
\begin{equation*}
\abs{\int_0^1 \omega_j(x+the_j) \, dt  - \omega(x)}
=
\abs{\int_0^1 \bigl( \omega_j(x+the_j) - \omega(x) \bigr) \, dt}
\leq
\epsilon .
\end{equation*}
That is,
\begin{equation*}
\lim_{h\to 0}\frac{f(x+h e_j) - f(x)}{h} = \omega_j(x) ,
\end{equation*}
which is what we wanted that is $df = \omega$.  As $\omega_j$ are
continuous for all $j$, we find that $f$ has continuous partial derivatives and
therefore is continuously differentiable.

For the other direction
suppose $f$ exists such that $df = \omega$.  Suppose we take a smooth
path
$\gamma \colon [a,b] \to U$ such that $\gamma(a) = x$ and
$\gamma(b) = y$, then
\begin{equation*}
\begin{split}
\int_\gamma df
& =
\int_a^b
\biggl(
\frac{\partial f}{\partial x_1}\bigl(\gamma(t)\bigr) \gamma_1^{\:\prime}(t)+
\frac{\partial f}{\partial x_2}\bigl(\gamma(t)\bigr) \gamma_2^{\:\prime}(t)+ \cdots +
\frac{\partial f}{\partial x_n}\bigl(\gamma(t)\bigr) \gamma_n^{\:\prime}(t)
\biggr) \, dt
\\
& = 
\int_a^b
\frac{d}{dt} \left[ f\bigl(\gamma(t)\bigr) \right]\, dt
\\
& = f(y)-f(x) .
\end{split}
\end{equation*}
The value of the integral only depends on $x$ and $y$, not the
path taken.  Therefore the integral is path independent.
We leave checking this for a piecewise smooth path as an exercise to the reader.
\end{proof}

\begin{prop}
Let $U \subset \R^n$ be a path connected open set and $\omega$
a 1-form defined on $U$.
Then $\omega = df$ for some continuously differentiable $f \colon U \to
\R$ if and only if
\begin{equation*}
\int_{\gamma} \omega = 0
\end{equation*}
for every piecewise smooth closed path $\gamma \colon [a,b] \to U$.
\end{prop}

\begin{proof}
Suppose first that $\omega = df$ and let $\gamma$ be a piecewise smooth
closed path.  Then we
from above we have that
\begin{equation*}
\int_{\gamma} \omega = f\bigl(\gamma(b)\bigr) - f\bigl(\gamma(a)\bigr) = 0 ,
\end{equation*}
because $\gamma(a) = \gamma(b)$ for a closed path.

Now suppose that for every piecewise smooth closed path $\gamma$, $\int_{\gamma} \omega = 0$.
Let $x,y$ be two points in $U$ and let $\alpha \colon [0,1] \to U$ and
$\beta \colon [0,1] \to U$ be two piecewise smooth paths with $\alpha(0) = \beta(0) = x$
and $\alpha(1) = \beta(1) = y$.  Then let $\gamma \colon [0,2] \to U$
be defined by
\begin{equation*}
\gamma(t) :=
\begin{cases}
\alpha(t) & \text{if $t \in [0,1]$,} \\
\beta(2-t) & \text{if $t \in (1,2]$.}
\end{cases}
\end{equation*}
This is a piecewise smooth closed path and so 
\begin{equation*}
0 = \int_{\gamma} \omega = \int_{\alpha} \omega - \int_{\beta} \omega .
\end{equation*}
This follows first by \propref{mv:prop:pathconcat}, and then noticing that
the second part is $\beta$ travelled backwards so that we get minus the
$\beta$ integral.  Thus the integral of $\omega$ on $U$ is path independent.
\end{proof}

There is a local criterion, a differential equation, that guarantees
path independence.  That is, under the right condition there exists
an \emph{antiderivative} $f$ whose total derivative is the given one-form
$\omega$.  However, since the criterion is local, we only get the result
locally.  We can define the antiderivative in any so-called
\emph{simply connected} domain, which informally is a domain where
any path between two points can be ``continuously deformed''
into any other path
between those two points.  To make matters simple, the usual way
this result is proved is for so-called star-shaped domains.

\begin{defn}
Let $U \subset \R^n$ be an open set and $p \in U$.  We say $U$ is
a \emph{\myindex{star shaped domain}}
with respect to $p$ if for any other point $x \in U$,
the line segment between $p$ and $x$ is in $U$, that is, if
$(1-t)p + tx \in U$ for all $t \in [0,1]$.
If we say simply \emph{star shaped}, then $U$ is star shaped with respect to
some $p \in U$.
\end{defn}

Notice the difference between star shaped and convex.  A convex domain is
star shaped, but a star shaped domain need not be convex.

\begin{thm}[\myindex{Poincar\`e lemma}]
Let $U \subset \R^n$ be a star shaped domain and $\omega$ a continuously
differentiable one-form defined on $U$.  That is, if
\begin{equation*}
\omega =
\omega_1 dx_1 +
\omega_2 dx_2 + \cdots +
\omega_n dx_n ,
\end{equation*}
then $\omega_1,\omega_2,\ldots,\omega_n$ are continuously differentiable
functions.  Suppose that for every $j$ and $k$
\begin{equation*}
\frac{\partial \omega_j}{\partial x_k} = \frac{\partial \omega_k}{\partial x_j} ,
\end{equation*}
then there exists a twice continuously differentiable function $f \colon U
\to \R$
such that $df = \omega$.
\end{thm}

The condition on the derivatives of $\omega$ is precisely the condition
that the second partial derivatives commute.  That is, if $df = \omega$,
and $f$ is twice continuously differentiable, then
\begin{equation*}
\frac{\partial \omega_j}{\partial x_k}
=
\frac{\partial^2 f}{\partial x_k \partial x_j} 
=
\frac{\partial^2 f}{\partial x_j \partial x_k} 
=
\frac{\partial \omega_k}{\partial x_j} .
\end{equation*}
The condition is therefore clearly necessary.  The lemma says that it is
sufficient for a star shaped $U$.

\begin{proof}
Suppose $U$ is star shaped with respect to $y=(y_1,y_2,\ldots,y_n) \in U$.

Given $x = (x_1,x_2,\ldots,x_n) \in U$, define the path $\gamma \colon [0,1] \to U$ as
$\gamma(t) := (1-t)y + tx$, so $\gamma^{\:\prime}(t) = x-y$.  Then let
\begin{equation*}
f(x) := \int_{\gamma} \omega
=
\int_0^1
\left(
\sum_{k=1}^n
\omega_k \bigl((1-t)y + tx \bigr) (x_k-y_k)
\right) \, dt .
\end{equation*}
We differentiate in $x_j$ under the integral.  We can do that since
everything, including the partials themselves are continuous.
\begin{equation*}
\begin{split}
\frac{\partial f}{\partial x_j}(x) & =
\int_0^1
\left(
\left(
\sum_{k=1}^n
\frac{\partial \omega_k}{\partial x_j} \bigl((1-t)y + tx \bigr) t
(x_k-y_k)
\right)
+
\omega_j \bigl((1-t)y + tx \bigr)
\right)
 \, dt
\\
& = 
\int_0^1
\left(
\left(
\sum_{k=1}^n
\frac{\partial \omega_j}{\partial x_k} \bigl((1-t)y + tx \bigr) t
(x_k-y_k)
\right)
+
\omega_j \bigl((1-t)y + tx \bigr)
\right) \, dt
\\
& = 
\int_0^1
\frac{d}{dt}
\left[
t \omega_j\bigl((1-t)y + tx \bigr)
\right]
\,
dt
\\
&= \omega_j(x) .
\end{split}
\end{equation*}
And this is precisely what we wanted.
\end{proof}

\begin{example}
Without some hypothesis on $U$ the theorem is not true.  Let
\begin{equation*}
\omega(x,y) := \frac{-y}{x^2+y^2} dx + \frac{x}{x^2+y^2} dy
\end{equation*}
be defined on $\R^2 \setminus \{ 0 \}$.  It is easy to see that
\begin{equation*}
\frac{\partial}{\partial y} \left[ 
\frac{-y}{x^2+y^2} \right] =
\frac{\partial}{\partial x} \left[ 
\frac{x}{x^2+y^2} \right] .
\end{equation*}
However, there is no $f \colon \R^2 \setminus \{ 0 \} \to \R$ such that 
$df = \omega$.  We saw in 
\exampleref{example:mv:irrotoneformint} if we integrate from $(1,0)$ to $(1,0)$
along the unit circle, that is $\gamma(t) = \bigl(\cos(t),\sin(t)\bigr)$
for $t \in [0,2\pi]$ we got $2\pi$ and not 0 as it should be if the integral
is path independent or in other words if there would exist an $f$ such that
$df = \omega$.
\end{example}

\subsection{Vector fields}

A common object to integrate is a so-called vector field.  That is an
assignment of a vector at each point of a domain.

\begin{defn}
Let $U \subset \R^n$ be a set.
A continuous function $v \colon U \to \R^n$ is called a
\emph{\myindex{vector field}}.  Write $v = (v_1,v_2,\ldots,v_n)$.

Given a smooth path $\gamma \colon [a,b] \to \R^n$ with
$\gamma\bigl([a,b]\bigr) \subset U$ we define
the path integral of the vectorfield $v$ as
\begin{equation*}
\int_{\gamma} v \cdot d\gamma
:=
\int_a^b v\bigl(\gamma(t)\bigr) \cdot \gamma^{\:\prime}(t) \, dt ,
\end{equation*}
where the dot in the definition is the standard dot product.
Again the definition of a piecewise smooth path is done by integrating over
each smooth interval and adding the result.
\end{defn}

If we unravel the definition we find that
\begin{equation*}
\int_{\gamma} v \cdot d\gamma
=
\int_{\gamma} v_1 dx_1 + v_2 dx_2 + \cdots + v_n dx_n .
\end{equation*}
Therefore what we know about integration of
one-forms carries over to the integration of vector fields.
For example path independence for integration of vector fields is simply
that
\begin{equation*}
\int_x^y v \cdot d\gamma
\end{equation*}
is path independent (so for any $\gamma$) if and only if 
$v = \nabla f$, that is the gradient of a function.  The function $f$
is then called the \emph{\myindex{potential}} for $v$.

A vector field $v$ whose path integrals are path independent is called
a \emph{\myindex{conservative vector field}}.  The naming comes from the
fact that such vector fields arise in physical systems
where a certain quantity, the energy is conserved.

\subsection{Exercises}

\begin{exercise}
Find an $f \colon \R^2 \to \R$ such that $df = xe^{x^2+y^2} dx +
ye^{x^2+y^2} dy$.
\end{exercise}

\begin{exercise}
Find an $\omega_2 \colon \R^2 \to \R$ such that
there exists a continuously differentiable $f \colon \R^2 \to \R$
for which
$df = e^{xy} dx + \omega_2 dy$.
\end{exercise}

\begin{exercise}
Finish the proof of \propref{mv:prop:pathinddf}, that is, we only proved the
second direction for a smooth path, not a piecewise smooth path.
\end{exercise}

\begin{exercise}
Show that a star shaped domain $U \subset \R^n$ is path connected.
\end{exercise}

\begin{exercise}
Show that $U := \R^2 \setminus \{ (x,y) \in \R^2 : x \leq 0, y=0 \}$ is
star shaped and find all points $(x_0,y_0) \in U$ such that
$U$ is star shaped with respect to $(x_0,y_0)$.
\end{exercise}

\begin{exercise}
Suppose $U_1$ and $U_2$ are two open sets in $\R^n$ with $U_1 \cap U_2$
nonempty and connected.
Suppose there exists an $f_1 \colon U_1 \to \R$ and
$f_2 \colon U_2 \to \R$, both twice continuously differentiable
such that $d f_1 = d f_2$ on $U_1 \cap U_2$.
Then there exists a twice differentiable function $F \colon U_1 \cup U_2 \to
\R$ such that $dF = df_1$ on $U_1$ and $dF = df_2$ on $U_2$.
\end{exercise}

\begin{exercise}[Hard]
Let $\gamma \colon [a,b] \to \R^n$ be a simple nonclosed piecewise smooth
path (so $\gamma$
is one-to-one).  Suppose $\omega$ is a continuously differentiable
one-form defined on some open
set $V$ with $\gamma\bigl([a,b]\bigr) \subset V$ and
$\frac{\partial \omega_j}{\partial x_k} = \frac{\partial \omega_k}{\partial
x_j}$
for all $j$ and $k$.  Prove that there exists an open set $U$
with $\gamma\bigl([a,b]\bigr) \subset U \subset V$ and
a twice continuously differentiable function $f \colon U \to \R$
such that $df = \omega$.
\\
Hint 1: $\gamma\bigl([a,b]\bigr)$ is compact.\\
Hint 2: Show that you can cover the curve by finitely many balls in sequence
so that the $k$th ball only intersects the $(k-1)$th ball.\\
Hint 3: See previous exercise.
\end{exercise}

\begin{exercise}
a)
Show that a connected open set is path connected.  Hint: Start with two
points $x$ and $y$ in a connected set $U$, and let $U_x \subset U$ is the set of points that are
reachable by a path from $x$ and similarly for $U_y$.  Show that both sets
are open, since they are nonempty ($x \in U_x$ and $y \in U_y$) it must be
that $U_x = U_y = U$.
\\
b) Prove the converse that is, a path connected set $U \subset \R^n$ is
connected.  Hint: for contradiction assume there exist two open and disjoint nonempty open
sets and then assume there is a piecewise smooth (and therefore continuous)
path between a point in one to a point in the other.
\end{exercise}

\begin{exercise}
Usually path connectedness is defined using just continuous paths rather
than piecewise smooth paths.  Prove that the definitions are equivalent, in
other words prove the following statement:\\
Suppose $U \subset \R^n$ is such that for any $x,y \in U$, there exists a continuous function
$\gamma \colon [a,b] \to U$ such that $\gamma(a) = x$ and $\gamma(b) = y$.
Then $U$ is path connected (in other words, then there exists a piecewise
smooth path).
\end{exercise}

\begin{exercise}[Hard]
Take
\begin{equation*}
\omega(x,y) = \frac{-y}{x^2+y^2} dx + \frac{x}{x^2+y^2} dy
\end{equation*}
defined on $\R^2 \setminus \{ (0,0) \}$.  Let $\gamma \colon [a,b] \to \R^2
\setminus \{ (0,0) \}$ be a closed piecewise smooth path.
Let $R:=\{ (x,y) \in \R^2 : x \leq 0 \text{ and } y=0 \}$.
Suppose $R \cap \gamma\bigl([a,b]\bigr)$ is a finite set of $k$ points.  Then
\begin{equation*}
\int_{\gamma} \omega = 2 \pi \ell 
\end{equation*}
for some integer $\ell$ with $\abs{\ell} \leq k$.\\
Hint 1: First prove that for a path $\beta$ that starts and end on $R$ but
does not intersect it otherwise, you find that $\int_{\beta} \omega$
is $-2\pi$, 0, or $2\pi$.
Hint 2: You proved above that $\R^2 \setminus R$ is star shaped.
\\
Note: The number $\ell$ is called the \emph{\myindex{winding number}} it measures how many
times does $\gamma$ wind around the origin in the clockwise direction.
\end{exercise}
